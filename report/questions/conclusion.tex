\section{Conclusion and possible improvements}
In conclusion, after exploring all kinds of different techniques to fit the wind power data correctly, we find that the one that produces the best results is the neural network. Indeed, it gave us the best scores while being a lot faster than stacking and random forests.
\\\\
We could improve our implementations and methods by fine-tuning the hyperparameters even more, preprocess the data in a better way. 
\\In the case of neural networks, finding the good selection of inputs to produce the best results could be a better approach (more inputs, outputs from previous computations of the power, ...). 
\\In the case of stacking, we could choose our regressors to cancel out the advantages and weaknesses of each regressor to produce a cleaner prediction.